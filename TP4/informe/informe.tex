\documentclass[12pt,a4paper]{article}
\usepackage[utf8]{inputenc}
\usepackage[spanish]{babel} % Para modificar labels por defecto
\usepackage[bottom]{footmisc} % Para las notas al pie de la página
\usepackage[T1]{fontenc}
\usepackage{siunitx}
\usepackage{tocbibind} % Bibliografía en el indice
\usepackage{titlesec} % Posibilidad de editar los formatos de chapter
\usepackage{amsmath,amssymb,mathrsfs} % Matemáticas varias
\usepackage{tabularx} % Para las tablas
\usepackage[title]{appendix} % Para anexos
\usepackage{fancyhdr}
\renewcommand{\baselinestretch}{1} % Interlineado. 1 es estandar
% --- Arreglos varios para la inclusion de imagenes
\usepackage{float}
\usepackage[pdftex]{graphicx}
\usepackage{subfigure}
\graphicspath{{/home/mrgreen/Pictures/}}
\usepackage[usenames,dvipsnames]{color}
\DeclareGraphicsExtensions{.png,.jpg,.pdf,.mps,.gif,.bmp}
% --- Para las dimensiones de los márgenes etc
\frenchspacing \addtolength{\hoffset}{-1.5cm}
\addtolength{\textwidth}{3cm} \addtolength{\voffset}{-2.5cm}
\addtolength{\textheight}{4cm}
% --- Para el encabezado
\setlength{\headheight}{33pt}

\fancyhead[R]{\includegraphics[height=1cm]{logo_fcefyn_nuevo.jpg}}
\fancyhead[L]{\includegraphics[height=1cm]{unc1_a.jpg}}
\fancyhead[C]{}
\fancyfoot[C]{Página \thepage} \renewcommand{\footrulewidth}{0.4pt}
\pagestyle{fancy}
% --- Para las tablas
\usepackage{multirow} % Juntar filas
\newcolumntype{L}[1]{>{\raggedright\arraybackslash}p{#1}} % Justificar Izq
\newcolumntype{C}[1]{>{\centering\arraybackslash}p{#1}} % Justificar Centrar
\newcolumntype{R}[1]{>{\raggedleft\arraybackslash}p{#1}} % Justificar Der
\usepackage[numbered]{bookmark} % Para que figure las secciones en el PDF
\usepackage{listings} % Para poner código
\usepackage{enumitem} % Para editar las propiedades de los items
\lstset{frame=single} % Código en un cuadro
\definecolor{negrou}{gray}{0.15} % Color gris para el fondo de las terminales
\lstset{basicstyle=\color{white}, backgroundcolor=\color{negrou}} % Código en un cuadro
% --- Para Anexo
\addto\captionsspanish{%
  \renewcommand\appendixname{ANEXO}
  \renewcommand\appendixpagename{ANEXOS}
  \renewcommand\tablename{Tabla}
}
% --- Quita los cuadrados de colores en los hipervínculos
\hypersetup{
	colorlinks = true,
	linkcolor  = black,
	urlcolor   = black,
  citecolor  = black,
  filecolor  = black,
  anchorcolor = black}
% -------------------------------------------------------- %
% Definicion de colores para el codigo

\lstdefinelanguage{Ce}{
  backgroundcolor=\color{white},		%color de fondo
  commentstyle=\color{CadetBlue},		%color de los comentarios
  keywordstyle=\color{Plum},			%color de las palabras reservadas
  numberstyle=\tiny\color{ForestGreen},	%numero de línea con tamaño y color
  numbers=left,                    	%posicion de los numeros
  numbersep=5pt,   					%espaciado entre los numeros
  stringstyle=\color{OliveGreen},		%color de las strings
  basicstyle=\ttfamily\footnotesize,	%estilo de fuente
  breakatwhitespace=false,         	%romper estacios en blanco
  breaklines=true,                 	%romper lineas
  frame=single,						%para encuadrar el codigo
  rulecolor=\color{black},			%color del cuadro
  showspaces=false,
  showstringspaces=false,
  showtabs=false,
  tabsize=2,
  language=C
}

\renewcommand{\lstlistingname}{Código}

% -------------------------------------------------------- %

\begin{document}

\begin{titlepage}
    \begin{center}
      \vspace*{1cm}

      \vspace{2cm}
      \includegraphics[width=0.4\textwidth]{unc1_a.jpg}
      \includegraphics[width=0.4\textwidth]{logo_fcefyn_nuevo.jpg}

      \Huge
      \textbf{Cátedra de Sistemas Operativos II}

      \vspace{3.5cm}

      \textbf{Trabajo Práctico N\si{\degree} III}

      \vfill

      \vspace{0.8cm}



      \Large
      Piñero, Tomás Santiago\\
      25 de Junio de 2020
    \end{center}
\end{titlepage}
% -------------------------------------------------------- %
% --- Tabla de contenidos

\setcounter{secnumdepth}{3}
\setcounter{tocdepth}{5}
\tableofcontents

% -------------------------------------------------------- %

\newpage
\renewcommand{\baselinestretch}{1}
\setlength{\parskip}{0.5em}

\section{Introducción}
\label{intro}

\subsection{Objetivo}
\label{objetivo}
El objetivo del trabajo es diseñar, crear, comprobar y validar una
aplicación de tiempo real sobre un RTOS.


\subsection{Definiciones, Acrónimos y Abreviaturas}
\label{daa}

\begin{itemize}
  \item \emph{RTOS}: \emph{Real Timer Operative System}. Sistema operativo en
  tiempo real.
  \item \emph{FreeRTOS}: es un sistema operativo en tiempo real para
  microcontroladores y pequeños microprocesadores.
  \cite{freertos}
  \item \emph{qemu}: es un emulador y virtualizador de procesadores genérico.
  \cite{qemu}
  \item \emph{UART}: Transmisor-receptor universal asíncrono.
\end{itemize}


\section{Descripción general}
\label{desc}

\subsection{Esquema del proyecto}
\label{esq}

El proyecto está dividido en las siguientes carpetas:

\begin{itemize}[leftmargin=1.5cm]
  \item \textbf{FreeRTOS}: contiene los archivos necesarios para la ejecución
  de \emph{FreeRTOS}. Los archivos fuentes de este último se encuentran en el
  subdirectorio \verb|Source|, mientras que los archivos del trabajo práctico
  se encuentran en el subdirectorio \verb|Demo/src|.

  \item \textbf{informe}: contiene los archivos correspondientes al informe.
\end{itemize}

\subsection{Requisitos futuros}
\label{futurereq}

\begin{itemize}[leftmargin=1.5cm]
  \item Mejorar la estructura de los directorios.
  \item Realizar el trabajo práctico en una \verb|LPC769|.
\end{itemize}

\newpage
\section{Diseño de solución}
\label{solucion}
Como se mencionó en la sección anterior, se organizó el proyecto en varias
carpetas (ver \textbf{\ref{esq}}).

Para realizar el práctico primero se consultó la documentación de \emph{FreeRTOS}
\cite{freedoc} y el libro \emph{Mastering the FreeRTOS Real Time Kernel,
A hands-on tutorial guide} \cite{freebook}. Consecuentemente, se decidió
utilizar el manejo de memoria dinámico \verb|heap_4.c|\cite{heap4}, que utiliza
el método \emph{Firs Fit} para la asignación y es el recomendado.

La comunicación de las tareas se realiza mediante colas\cite{queue}, por lo que
habrá una única cola en todo el sistema.

\subsection{Alnálisis del \emph{scheduler}}
\label{sched}
La documentación de \emph{FreeRTOS} explica que \emph{scheduler} siempre elige
la tarea con mayor prioridad que esté lista para ejecutarse (\emph{Ready state})
y, en caso de existir dos tareas con la misma prioridad, utiliza el método
\emph{Round Robin} alternando entre ellas.

Para verificar esto, se utilizó la tarea \emph{top} que muestra el estado del
sistema con la misma prioridad que la tarea con máxaima prioridad. En el caso de
este trabajo, las tareas no abandonan el procesador por sí solas, realizando
que el \emph{scheduler} funcione simplemente por nivel de prioridad.

A continuación, se muestra el estado del sistema con el consumidor con mayor
prioridad sobre los productores:

\begin{figure}[H]
  \centering
  \includegraphics[scale=0.7]{Cmax.png}
  \caption{Ejecución del programa con prioridad máxima (4) para C1.}
  \label{cmax}
\end{figure}

\newpage
Como puede observarse en la Fig. \textbf{\ref{cmax}}, el consumidor ocupa todo
el tiempo del procesador, debido a que la tarea siempre está lista para
ejecutarse y tiene la mayor prioridad, produciéndose así la inanición a los
productores.

En caso de tener todas las tareas la misma prioridad, el \emph{scheduler} reparte
el tiempo del procesador entre los procesos con el método \emph{Round Robin}.

\begin{figure}[H]
  \centering
  \includegraphics[scale=0.7]{Equals.png}
  \caption{Ejecución del programa con todas las tareas con la misma prioridad.}
  \label{equals}
\end{figure}

Como se puede observar en la Fig. \textbf{\ref{equals}}, todas las tareas
tuvieron un mismo tiempo de ejecución debido a que todas ellas comparten la
misma prioridad.

\subsection{\emph{Stack} mínimo para cada tarea}
\label{minstack}
Mediante la función \verb|uxTaskGetStackHighWatermark| se puede conocer la
cantidad de espacio de la pila que no fue utilizado durante la ejecución de la
tarea. Este valor se encuentra expresado en palabras (\emph{words})\cite{water},
por lo que en un sistema de 32 bits, una palabra significan 4 bytes de pila no
utilizada.

Se iniciará con un valor de 280 bytes (70 en palabras) para cada tarea y se
utilizará la función para ajustarlo según la tarea.

\subsection{\emph{Top Task}}
\label{toptask}
La tarea tipo \emph{top} de \emph{Linux} puede realizarse con la función
\verb|vTaskGetRunTimeStats|, que genera y muestra una tabla con la información
sobre el uso de la CPU de cada tarea. Esta tarea se ejecutará cada 5 segundos,
pero este tiempo puede ser modificado al momento de ejecución modificando la
variable \verb|mainCHECK_DELAY|, cuyo valor se encuentra expresado en
milisegundos.


\section{Implementación y resultados}
\label{impres}
Toda muestra de información se realiza por medio del periférico \emph{UART},
que se muestra en consola utilizando la \emph{flag} \verb|-serial mon:stdio| de
\emph{qemu}.

Se utilizó la prioridad máxima por defecto \verb|configMAX_PRIORITIES|, cuyo
valor es 5 y es utilizada como base para la prioridad de todas las tareas, por lo
que la prioridad de la tarea puede ir de 0 a 4.

\subsection{Productor}
\label{prodtask}
Los productores utilizan el Código \textbf{\ref{ptask}}, en el que simplemente
produce (pone en la cola) su nombre. Su prioridad es 2
(\verb|configMAX_PRIORITIES-3|).

\begin{lstlisting}[caption={Función `\emph{vProducerTask}'.}, label={ptask}, language=Ce]
static void vProducerTask( void *pvParameters )
{
  UBaseType_t uxHighWaterMark;
	uxHighWaterMark = uxTaskGetStackHighWaterMark( NULL );
	char *pcMessage = pcTaskGetName( NULL );

	for( ;; )
	{
		/* Send message to the queue. */
		if(xQueueSend( xPrintQueue, &pcMessage, 0 ) != pdTRUE)
		{
			;
			// vUARTPrint(pcMessage);
			// vUARTPrint(" cannot send message.\n\r");
		}

		uxHighWaterMark = uxTaskGetStackHighWaterMark( NULL );
	}
}
\end{lstlisting}

\subsection{Consumidor}
\label{contask}
El consumidor muestra qué valor de la cola ha consumido cada 1 segundo con el
siguiente formato:

\begin{center}
\verb|C1 -> `nombreProductor'|
\end{center}

Su código es el siguiente:

\begin{lstlisting}[caption={Función `\emph{vConsumerTask}'.}, label={ctask}, language=Ce]
static void vConsumerTask( void *pvParameters )
{
  UBaseType_t xStatus;
	char *pcMessage;
	char *pcTaskName = pcTaskGetName( NULL );


	for( ;; )
	{
		// vTaskDelay(1000);
		/* Wait for a message to arrive. */
		if(xQueueReceive( xPrintQueue, &pcMessage, 0 ) == pdFALSE)
		{
			;//vUARTPrint("Empty queue.\n\r"); portMAX_DELAY
		}
		else
		{
			/* Print Consumer name -> Producer name */
			vUARTPrint(pcTaskName);
			vUARTPrint(" -> ");
			vUARTPrint(pcMessage);
			vUARTPrint("\r\n");
		}
	}
}
\end{lstlisting}

\subsection{\emph{Top}}
\label{totask}
La tarea top se llevo a cabo utilizando las funciones \emph{vTaskList} y
\emph{vTaskGetRunTimeStats} provistas por \emph{FreeRTOS}. La primera muestra
el estado de las tareas del sistema con el siguiente formato:

\begin{center}
\verb|Tarea    Estado    Prioridad    Stack    Numero de tarea|
\end{center}

La segunda función, muestra el tiempo que cada tarea estuvo utilizando el
procesador, tanto en valor absoluto como en porcentaje, de la siguiente forma:

\begin{center}
\verb|Tarea    ABS    %CPU|
\end{center}

La tarea \emph{top} se ejecuta cada 5 segundos y su código es el siguiente:

\begin{lstlisting}[caption={Función `\emph{vTopTask}'.}, label={run}, language=Ce]
static void vTopTask( void *pvParameters )
{
	TickType_t xLastExecutionTime;
	char *buffer = pvPortMalloc(sizeof(char));

	/* Initialise xLastExecutionTime so the first call to vTaskDelayUntil()
	works correctly. */
	xLastExecutionTime = xTaskGetTickCount();

	for( ;; )
	{
		/* Perform this check every mainCHECK_DELAY milliseconds. */
		vTaskDelayUntil( &xLastExecutionTime, mainCHECK_DELAY );


		/* Imprimir 'top' cada 5 segundos */
		vUARTPrint("\n----------------Runtime Stats-----------------\n\r");
		vUARTPrint("Task\t\tABS\t\t%CPU\n\r");
		vUARTPrint("----------------------------------------------\n\r");

		vTaskGetRunTimeStats(buffer);
		vUARTPrint(buffer);
		vUARTPrint("----------------------------------------------\n\r");

		vUARTPrint("\n------------------Task State------------------\n\r");
		vUARTPrint("Task\t\tState\tPrio\tStack\tNum\n\r");
		vUARTPrint("----------------------------------------------\n\r");
		vTaskList(buffer);
		vUARTPrint(buffer);
		vUARTPrint("----------------------------------------------\n\r");
	}
}

\end{lstlisting}


\subsection{Resultados}
\label{results}

\begin{figure}[H]
  \centering
  \includegraphics[scale=0.5]{restp4.png}
  \caption{Ejecución del programa.}
  \label{exec}
\end{figure}

La Fig. \textbf{\ref{exec}} muestra la ejecución del programa, donde se puede
observar el orden en que los productores lograron acceder a la cola.
Pasados los 5 segundos, se imprimen las estadísticas de las tareas del sistema.
La única ejecutándose es la tarea \emph{top}, ya que para obtener estas
estadísticas, \emph{FreeRTOS} bloquea la tarea Consumidor, que tiene la misma
prioridad, mientras que las demás están listas para ejecutarse.

\section{Conclusiones}
\label{conc}
Utilizando la función \verb|uxTaskGetStackHighWatermark| y la tarea \emph{top}
se pudo definir el tamaño de pila necesario para cada tarea, si bien la
documentación oficial recomienda no cambiar el valor por defecto, los cambios
realizados no afectaron el funcionamiento del programa.

Respecto al trabajo práctico, lo que más dificultad generó fue la tarea \emph{top},
dado que las funciones leídas de la documentación no funcionaban como uno
esperaba. Su implementación llevó tiempo debido a esto y a que se tuvo que
llegar a conocer la placa con la que se estaba trabajando.

\newpage
\begin{appendices}
\section{Guía paso a paso}
\label{app:howto}

Para poder ejecutar el programa deben ejecutarse los siguientes comandos
(la ejecución se detiene presionando \texttt{Ctrl+A X}):

\subsection{\emph{Github}}
\label{git}

\begin{lstlisting}[breaklines,language=bash,basicstyle=\ttfamily\color{white}]
 $ git clone https://github.com/Sistemas-Operativos-II-2020/so2-tp4-rtos-tomasspi.git
 $ cd so2-tp1-rtos-tomasspi/FreeRTOS/Demo/src
 $ make
 $ qemu-system-arm -machine lm3s811evb -cpu cortex-m3 -nographic -serial mon:stdio -kernel gcc/RTOSDemo.axf -gdb tcp::3333
\end{lstlisting}

\subsection{Archivo comprimido}
\label{tar}

Cambiar la letra n por la ñ en `Pinero'.

\begin{lstlisting}[breaklines,language=bash,basicstyle=\ttfamily\color{white}]
 $ tar -zxvf TP4_TPinero_39445871.tar.gz
 $ cd TP4_TPinero_39445871/FreeRTOS/Demo/src
 $ make
 $ qemu-system-arm -machine lm3s811evb -cpu cortex-m3 -nographic -serial mon:stdio -kernel gcc/RTOSDemo.axf -gdb tcp::3333
\end{lstlisting}

\section{\emph{Debugging}}
\label{debug}
Para realizar un \emph{debugging} se deben ingresar los siguientes comandos
desde el directorio \verb|src|:

\begin{lstlisting}[breaklines,language=bash,basicstyle=\ttfamily\color{white}]
 $ gdb-multiarch gcc/RTOSDemo.axf
 $ target remote localhost:3333
\end{lstlisting}

A partir de ese momento, pueden utilizarse los comandos de \emph{gdb}\cite{cheat}.
\end{appendices}

\begin{thebibliography}{9}
\bibitem{freertos}
\emph{FreeRTOS},
\\\texttt{\href{https://www.freertos.org/}{Link to FreeRTOS}}

\bibitem{freedoc}
\emph{FreeRTOS, API Reference},
\\\texttt{\href{https://www.freertos.org/a00106.html}{Link to FreeRTOS API Reference}}

\bibitem{freebook}
\emph{Richard Barry, Mastering the FreeRTOS Real Time Kernel, A hands-on tutorial guide},
\\\texttt{\href{https://www.freertos.org/wp-content/uploads/2018/07/161204_Mastering_the_FreeRTOS_Real_Time_Kernel-A_Hands-On_Tutorial_Guide.pdf}{Link to Mastering FreeRTOS}}

\bibitem{heap4}
\emph{FreeRTOS, Memory Management},
\\\texttt{\href{https://www.freertos.org/a00111.html}{Link to Memory Management}}

\bibitem{queue}
\emph{FreeRTOS, Queues},
\\\texttt{\href{https://www.freertos.org/Embedded-RTOS-Queues.html}{Link to Queues}}

\bibitem{water}
\emph{FreeRTOS, Documentation},
\\\texttt{\href{https://www.freertos.org/uxTaskGetStackHighWaterMark.html}{Link to uxTaskGetStackHighWaterMark}}

\bibitem{state}
\emph{FreeRTOS, Documentation},
\\\texttt{\href{https://www.freertos.org/uxTaskGetSystemState.html}{Link to uxTaskGetSystemState}}

\bibitem{qemu}
\emph{qemu},
\\\texttt{\href{https://www.qemu.org/}{Link to qemu}}

\bibitem{cheat}
\emph{rkubik, gdb cheatsheet},
\\\texttt{\href{https://gist.github.com/rkubik/b96c23bd8ed58333de37f2b8cd052c30}{Link to gdb cheatsheet}}
\end{thebibliography}
\end{document}
