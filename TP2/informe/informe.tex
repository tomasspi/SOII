\documentclass[12pt,a4paper]{article}
\usepackage[utf8]{inputenc}
\usepackage[spanish]{babel} % Para modificar labels por defecto
\usepackage[bottom]{footmisc} % Para las notas al pie de la página
\usepackage[T1]{fontenc}
\usepackage{siunitx}
\usepackage{tocbibind} % Bibliografía en el indice
\usepackage{titlesec} % Posibilidad de editar los formatos de chapter
\usepackage{amsmath,amssymb,mathrsfs} % Matemáticas varias
\usepackage{tabularx} % Para las tablas
\usepackage[title]{appendix} % Para anexos
\usepackage{fancyhdr}
\renewcommand{\baselinestretch}{1} % Interlineado. 1 es estandar
% --- Arreglos varios para la inclusion de imagenes
\usepackage{float}
\usepackage[pdftex]{graphicx}
\usepackage{subfigure}
\graphicspath{{/home/mrgreen/Pictures/}}
\usepackage[usenames,dvipsnames]{color}
\DeclareGraphicsExtensions{.png,.jpg,.pdf,.mps,.gif,.bmp}
% --- Para las dimensiones de los márgenes etc
\frenchspacing \addtolength{\hoffset}{-1.5cm}
\addtolength{\textwidth}{3cm} \addtolength{\voffset}{-2.5cm}
\addtolength{\textheight}{4cm}
% --- Para el encabezado
\setlength{\headheight}{33pt}

\fancyhead[R]{\includegraphics[height=1cm]{logo_fcefyn_nuevo.jpg}}
\fancyhead[L]{\includegraphics[height=1cm]{unc1_a.jpg}}
\fancyhead[C]{}
\fancyfoot[C]{Página \thepage} \renewcommand{\footrulewidth}{0.4pt}
\pagestyle{fancy}
% --- Para las tablas
\usepackage{multirow} % Juntar filas
\newcolumntype{L}[1]{>{\raggedright\arraybackslash}p{#1}} % Justificar Izq
\newcolumntype{C}[1]{>{\centering\arraybackslash}p{#1}} % Justificar Centrar
\newcolumntype{R}[1]{>{\raggedleft\arraybackslash}p{#1}} % Justificar Der
\usepackage[numbered]{bookmark} % Para que figure las secciones en el PDF
\usepackage{listings} % Para poner código
\usepackage{enumitem} % Para editar las propiedades de los items
\lstset{frame=single} % Código en un cuadro
\definecolor{negrou}{gray}{0.15} % Color gris para el fondo de las terminales
\lstset{basicstyle=\color{white}, backgroundcolor=\color{negrou}} % Código en un cuadro
% --- Para Anexo
\addto\captionsspanish{%
  \renewcommand\appendixname{ANEXO}
  \renewcommand\appendixpagename{ANEXOS}
  \renewcommand\tablename{Tabla}
}
% --- Quita los cuadrados de colores en los hipervínculos
\hypersetup{
	colorlinks = true,
	linkcolor  = black,
	urlcolor   = black,
  citecolor  = black,
  filecolor  = black,
  anchorcolor = black}
% -------------------------------------------------------- %
% Definicion de colores para el codigo

\lstdefinelanguage{Ce}{
  backgroundcolor=\color{white},		%color de fondo
  commentstyle=\color{CadetBlue},		%color de los comentarios
  keywordstyle=\color{Plum},			%color de las palabras reservadas
  numberstyle=\tiny\color{ForestGreen},	%numero de línea con tamaño y color
  numbers=left,                    	%posicion de los numeros
  numbersep=5pt,   					%espaciado entre los numeros
  stringstyle=\color{OliveGreen},		%color de las strings
  basicstyle=\ttfamily\footnotesize,	%estilo de fuente
  breakatwhitespace=false,         	%romper estacios en blanco
  breaklines=true,                 	%romper lineas
  frame=single,						%para encuadrar el codigo
  rulecolor=\color{black},			%color del cuadro
  showspaces=false,
  showstringspaces=false,
  showtabs=false,
  tabsize=2,
  language=C
}

\lstdefinelanguage{mperl}
{
  backgroundcolor=\color{white},
  commentstyle=\color{CadetBlue},
  keywordstyle=\bfseries\color{Plum},
  identifierstyle=\color{MidnightBlue},
  stringstyle=\color{OliveGreen},
  numberstyle=\tiny\color{ForestGreen},
  numbers=left,
  numbersep=5pt,
  basicstyle=\ttfamily\footnotesize,
  alsoletter={\%},
  breakatwhitespace=false,
  breaklines=true,
  frame=single,
  rulecolor=\color{black},
  showspaces=false,
  showstringspaces=false,
  showtabs=false,
  tabsize=2,
  language=Perl
}

\renewcommand{\lstlistingname}{Código}

% -------------------------------------------------------- %

\begin{document}

\begin{titlepage}
    \begin{center}
      \vspace*{1cm}

      \vspace{2cm}
      \includegraphics[width=0.4\textwidth]{unc1_a.jpg}
      \includegraphics[width=0.4\textwidth]{logo_fcefyn_nuevo.jpg}

      \Huge
      \textbf{Cátedra de Sistemas Operativos II}

      \vspace{3.5cm}

      \textbf{Trabajo Práctico N\si{\degree} II}

      \vfill

      \vspace{0.8cm}



      \Large
      Piñero, Tomás Santiago\\
      14 de Mayo de 2020
    \end{center}
\end{titlepage}
% -------------------------------------------------------- %
% --- Tabla de contenidos

\setcounter{secnumdepth}{3}
\setcounter{tocdepth}{5}
\tableofcontents

% -------------------------------------------------------- %

\newpage
\renewcommand{\baselinestretch}{1}
\setlength{\parskip}{0.5em}

\section{Introducción}
\label{intro}

\subsection{Objetivo}
\label{objetivo}
El objetivo del trabajo es diseñar una solución que utilice el paradigma de
memoria compartida mediante \emph{OpenMP} para realizar el procesamiento de
una imagen \emph{BMP}.


\subsection{Definiciones, Acrónimos y Abreviaturas}
\label{daa}

\begin{itemize}
  \item \emph{BMP}: \emph{Bitmap}. Formato de imagen capaz de almacenar
  imágenes bidimensionales tanto en color como monocromático.\cite{bmpfor}
  \item \emph{API}: \emph{Application Programming Interface}. Es un conjunto
  de funciones ofrecidas para ser utilizadas por otro software.
  \item Nodo: computadora individual que forma parte de un \emph{cluster}.
  \item \emph{Cluster}: conjunto de nodos unidos entre sí que se comportan
  como una única computadora.
  \item \emph{OpenMP}: \emph{Open Multi-Processing}. Es una API para programación
  multiprocesos de memoria compartida.\cite{omp}
\end{itemize}


\section{Descripción general}
\label{desc}

\subsection{Restricciones}
\label{restrictions}
A las restricciones dadas (ver \href{run:../Enunciado.pdf}
{\texttt{Enunciado.pdf}}) se le agregaron las siguientes:

\begin{enumerate}[leftmargin=1.5cm]
  \item La imagen a filtrar debe estar ubicada en la carpeta \verb|imgs|.
  \item Para ejecutar el programa se debe ingresar desde la carpeta \verb|bin|:

  \begin{center}
    \verb|./main `nombre_imagen' `radio' `cantidad_hilos'|
  \end{center}
\end{enumerate}


\subsection{Esquema del proyecto}
\label{esq}

El proyecto está dividido en las siguientes carpetas:

\begin{itemize}[leftmargin=1.5cm]
  \item \textbf{bin}: contiene el archivo ejecutable.
  \item \textbf{inc}: contiene los \emph{headers} utilizados por el
  código fuente:

  \begin{itemize}[leftmargin=1cm, nosep]
    \item \emph{simple\_bmp.h}: \emph{header} para la carga, lectura y
    escritura de las imágenes \emph{BMP}.
  \end{itemize}

  \item \textbf{imgs}: contiene la imagen a la que se aplican los filtros y
  el resultado de su aplicación.
  \item \textbf{src}: contiene los códigos fuente.

  \begin{itemize}[leftmargin=1cm, nosep]
    \item \emph{main.c}: realiza la aplicación de los filtros a la imagen de
    entrada.
    \item \emph{simple\_bmp.c}: contiene la implementación de las funciones de
    \emph{simple\_bmp.h}.
  \end{itemize}
\end{itemize}

\subsection{Requisitos futuros}
\label{futurereq}

\begin{itemize}[leftmargin=1.5cm]
  \item Utilizar alguno de los métodos conocidos para el tratamiento de los
  bordes de la imagen.
  \item Soportar más de dos filtros de imagen.
\end{itemize}


\section{Diseño de solución}
\label{solucion}
Como se mencionó en la sección anterior, se organizó el proyecto en varias
carpetas (ver \textbf{\ref{esq}}).

Luego, se decidió por separar la aplicación de los filtros en dos funciones
distintas para una mejor legibilidad del código.
Por lo tanto, se creará una función que determine la posición del píxel
respecto al centro de la imagen y en función del resultado aplique el filtro
correspondiente.

En el caso de que el píxel corresponda a alguno de los bordes de la imagen,
no será modificado.

Los parámetros correspondientes a los filtros se establecen en el momento de
compilación, para evitar una pérdida de tiempo en la obtención de los
parámetros en el momento de la ejecución. Los parámetros con
los que se realizaron las ejecuciones son los siguientes:

\begin{itemize}[leftmargin=1.5cm]
  \item Brillo (\emph{L}): 1.8
  \item Contraste (\emph{K}): 15
  \item Radio (\emph{R}): 500 [px]
  \item Tamaño de kernel (\emph{SIZE\_K}): 45
\end{itemize}

\subsection{Profilers}
\label{prof}
Un \emph{profiler} es un \emph{software} que realiza un análisis sobre el
comportamiento de un determinado programa mediante la información obtenida
durante la ejecución del mismo.
Este tipo de herramienta es utilizado para depurar y optimizar los algoritmos
utilizados en el programa.

A continuación se muestran las herramientas de \emph{profiling} para analizar
la \emph{performance} del programa realizado.

\subsubsection{perf}
\label{perf}
Es una herramienta disponible para \emph{Linux} a partir de la version de
kernel \verb|2.6.31| (2009).

Soporta contadores de rendimiento tanto de \emph{hardware} como \emph{software},
puntos de rastreao y sondas dinámicas.
\newpage
Algunos de las opciones de ejecución son las siguientes:

\begin{itemize}[leftmargin=1.5cm]
  \item \emph{stat}: mide el recuento total de eventos.
  \item \emph{record}: mide y guarda los datos de muestreo.
  \item \emph{report}: analiza archivo generado por registro perf. Puede generar
  un \emph{profile} plano o un grafo de llamadas.
  \item \emph{sched}: seguimiento/medición de las acciones y latencias del
  \emph{scheduler}.
\end{itemize}

Para ejecutar \emph{perf} se debe realizar lo siguiente:

\begin{lstlisting}[breaklines,language=bash,basicstyle=\ttfamily\color{white}]
  $ sudo perf stat ./main `nombre_imagen' `radio' `cantidad_hilos'
\end{lstlisting}

\begin{figure}[H]
  \centering
  \includegraphics[scale=0.5]{perf.png}
  \caption{Resultado de \emph{perf stat}.}
  \label{outp}
\end{figure}

\subsubsection{gprof}
\label{gprof}
Es la mejora de \emph{perf}. Para poder utilizarla, el programa a analizar
debe compilarse con la opción \verb|-pg|, que genera archivo con un grafo de
llamadas (\emph{default} \verb|gmon.out|). Este archivo es leído por \emph{gprof}
y calcula los tiempos de cada función utilizada.

Para hacer un análisis con \emph{gprof} se deben ejecutar los siguientes comandos:

\begin{lstlisting}[breaklines,language=bash,basicstyle=\ttfamily\color{white}]
 $ ./main.o `nombre_imagen' `radio' `cantidad_hilos'
 $ gprof ./main.o gmon.out
\end{lstlisting}

\begin{figure}[H]
  \centering
  \includegraphics[scale=0.5]{flat.png}
  \caption{Salida plana de \emph{gprof}.}
  \label{goutf}
\end{figure}

Véase que la Figura \textbf{\ref{goutf}} indica que el 99.61\% del tiempo
de ejecución del programa pertenece a la función \emph{`blur\_filter'} (ver
\textbf{\ref{blur}}), por lo que se utilizará \emph{OpenMP} en la función
\emph{`apply\_filters'}.

\subsubsection{\emph{Valgrind}}
\label{valgrind}
Es un conjunto de herramientas que se utilizan para depurar problemas de
memoria y rendimiento de un programa.
Algunas de estas herramientas son:

\begin{itemize}[leftmargin=1.5cm, itemsep=1pt]
  \item \emph{memcheck}: realiza un seguimiento del uso de la memoria.
  \item \emph{cachegrind}: mide el rendimiento de la caché durante la ejecución.
  \item \emph{helgrind}: detecta las condiciones de carrera en un código multihilo.
\end{itemize}

Con \emph{Valgrind}, el programa a \emph{testear} se ejecuta entre cinco y veinte
veces más lento, por lo que es ralmente importante tenerlo en cuenta al momento
de su uso.

\begin{figure}[H]
  \centering
  \includegraphics[scale=0.5]{valgrind.png}
  \caption{Resultado de \emph{valgrind} con la herramienta \emph{cachegrind}.}
  \label{cachegrind}
\end{figure}

\subsubsection{\emph{V-Tune}}
\label{vtune}
Es desarrollado por Intel. Puede utilizarse con una interfaz gráfica o por medio de
la línea de comandos. Los comandos soportados pueden verse en el siguiente enlace:
\href{https://software.intel.com/content/www/us/en/develop/documentation/vtune-help/top.html}
{\emph{user guide}}.

\begin{figure}[H]
  \centering
  \includegraphics[scale=0.5]{hotspot.png}
  \caption{Resultado de \emph{vtune} con la herramienta \emph{hotspots}.}
  \label{hotspot}
\end{figure}

\subsection{\emph{Makefile}}
\label{make}
Las recetas disponibles en el \emph{Makefile} son las siguientes:

\begin{enumerate}[leftmargin=1.5cm]
  \item \emph{all}: genera el ejecutable del sistema.
  \item \emph{check}: corre \emph{cppcheck} sobre el directorio
  \verb|src|.
  \item \emph{doc}: genera la documentación del código en HTML
  utilizando \emph{doxygen} en la carpeta llamada \verb|Doc| y la
  abre en el navegador \emph{Firefox}.\footnote{Si no se encuentra
  instalado, se sugiere cambiar el navegador en la receta.}
  \item \emph{clean}: elimina los directorios \verb|bin| y
  \verb|Doc|.
\end{enumerate}


\section{Implementación y resultados}
\label{impres}
En las siguientes subsecciones se explicarán las funciones encargadas de aplicar
los filtros y luego los resultados obtenidos con distintas configuraciones.

Para reducir los tiempos se declararon ambas imágenes (de entrada y salida) a
nivel global, evitando una sobrecarga en la pila y logrando un acceso más rápido
a estas dos variables\cite{global}.

\subsection{Aplicación de los filtros}

\begin{lstlisting}[caption={Función `\emph{apply\_filters}'.}, label={apply}, language=Ce]
int32_t distancia = 0;
int32_t distancia_y = 0;
int32_t radio_cuadrado = r*r;

uint16_t **kernel = calloc(SIZE_K, sizeof (int *));

for(int k = 0; k < SIZE_K; k++)
  kernel[k] = calloc(SIZE_K, sizeof (uint16_t));

kernel_setup(kernel, SIZE_K);

uint16_t norm = get_norm(kernel, SIZE_K);
int32_t offset = (int32_t) ((SIZE_K-1) / 2);

double start = omp_get_wtime();

#pragma omp parallel num_threads(threads)
#pragma omp for schedule(dynamic)
for (int32_t y = offset; y < in.info.image_height-offset; y++)
{
  distancia_y = (y - c.pos_y)*(y - c.pos_y);

  for (int32_t x = offset; x < in.info.image_width-offset; x++)
  {
    distancia = (x - c.pos_x)*(x - c.pos_x) + distancia_y;

    if(distancia <= radio_cuadrado) /** Inside circle */
      out.data[y][x] = lineal_filter(in.data[y][x]);
    else /* Outside circle */
      out.data[y][x] = blur_filter(x, y, kernel, norm);
  }
}

printf("\n-- Tiempo transcurrido (%d hilos): %f --\n\n", threads,
       omp_get_wtime() - start);
\end{lstlisting}

El Código \textbf{\ref{apply}} muestra la implementación de la función que
aplica ambos filtros en la imagen. Dicha función recorre la imagen por filas y
determina si el píxel está o no dentro de la circunferencia de radio especificado.
En caso de estar dentro, se aplica el filtro lineal, sino se aplica el de
desenfoque.


\subsubsection{Filtro lineal}
\label{msg_h}

\begin{lstlisting}[caption={Función `\emph{lineal\_filter}'.}, label={lineal}, language=Ce]
int32_t r, g, b;
r = pixel.red;
g = pixel.green;
b = pixel.blue;

r = (int32_t) (r * K) + L;
r = normalize_pixel_color(r);

g = (int32_t) (g * K) + L;
g = normalize_pixel_color(g);

b = (int32_t) (b * K) + L;
b = normalize_pixel_color(b);

pixel = (sbmp_raw_data) { (uint8_t) b, (uint8_t) g, (uint8_t) r};

return pixel;
\end{lstlisting}

El Código \textbf{\ref{lineal}} muestra la implementación de la función
encargada de aplicar el filtro lineal dentro de la circunferencia dada. Esta
función toma cada uno de los colores del píxel, los modifica según los valores
de brillo y contraste dados, y devuelve el píxel modificado.

\subsubsection{Filtro de desenfoque}
\label{blurf}

\begin{lstlisting}[caption={Función `\emph{blur\_filter}'.}, label={blur}, language=Ce]
int32_t sum_r = 0;
int32_t sum_g = 0;
int32_t sum_b = 0;
int32_t offset = (int32_t) ((SIZE_K-1) / 2);

for (int32_t m = -offset; m <= offset; m++)
  {
    for (int32_t n = -offset; n <= offset; n++)
      {
        sum_r += (int32_t) (kernel[m+offset][n+offset] * in.data[y+m][x+n].red);
        sum_g += (int32_t) (kernel[m+offset][n+offset] * in.data[y+m][x+n].green);
        sum_b += (int32_t) (kernel[m+offset][n+offset] * in.data[y+m][x+n].blue);
      }
  }
sum_r = (int32_t) (sum_r / norm);
sum_g = (int32_t) (sum_g / norm);
sum_b = (int32_t) (sum_b / norm);

return (sbmp_raw_data) {(uint8_t) sum_b, (uint8_t) sum_g, (uint8_t) sum_r};
\end{lstlisting}

El Código \textbf{\ref{blur}} muestra la implementación de la función
encargada de aplicar el filtro de desenfoque fuera de la circunferencia dada.
Simplemente realiza la convolución entre la matriz kernel y la imagen de entrada,
una vez por cada color, los normaliza y los devuelve en el nuevo píxel.


\subsection{Resultados}
\label{resultados}

\subsubsection{Filtrado}
\label{filtrado}
A continuación se muestran los resultados de la modificación de la imagen dada
y dos a elección del estudiante con los parámetros mencionados la
sección \textbf{\ref{solucion}}.

\begin{figure}[H]
  \centering
  \includegraphics[width=10cm]{in1.png}
  \caption{Imagen de entrada.}
  \label{input}
\end{figure}

\begin{figure}[H]
  \centering
  \includegraphics[width=10cm]{out1.png}
  \caption{Imagen de salida.}
  \label{output}
\end{figure}

\begin{figure}[H]
  \centering
  \includegraphics[width=10cm]{in2.png}
  \caption{Imagen de entrada.}
  \label{input1}
\end{figure}

\begin{figure}[H]
  \centering
  \includegraphics[width=10cm]{out2.png}
  \caption{Imagen de salida.}
  \label{output1}
\end{figure}

\begin{figure}[H]
  \centering
  \includegraphics[width=10cm]{in3.png}
  \caption{Imagen de entrada.}
  \label{input2}
\end{figure}

\begin{figure}[H]
  \centering
  \includegraphics[width=10cm]{out3.png}
  \caption{Imagen de salida.}
  \label{output2}
\end{figure}

\subsubsection{Medición de tiempos}
\label{tiempos}
Para medir los tiempos de ejecución se realizó un \emph{script} en \emph{pearl}.

\begin{lstlisting}[caption={\emph{Script} de ejecución.}, label={script}, language=mperl]
#!/usr/bin/perl

use Time::HiRes qw[time];

$M = 1;
$muestras = $M;
$pot = 0;
$hilos = 1;
$max_hilos = 16;
$radio = 100;

while($hilos <= $max_hilos)
{
	while($muestras > 0)
	{
		$cmd = "./main base.bmp $radio $hilos";
		system($cmd);
		$muestras--;
	}
	$muestras = $M;
	$pot++;
    $hilos = 2**$pot;
}

print("DONE.\n");
\end{lstlisting}

El Código \textbf{\ref{script}} muestra dicho \emph{script}, que realiza una
determinada cantidad de muestras para una cantidad de hilos que varía en
potencias de 2 hasta alcanzar una cantidad total de 64.

Este \emph{script} se ejecuta tanto en la PC local como en un nodo del
\emph{cluster} de la Facultad, \emph{Wayra}.
Dicho nodo tiene un procesador \verb|Intel Xeon E5-2630 v4|
\footnote{Para información detallada del procesador, dirigirse
\href{https://ark.intel.com/content/www/es/es/ark/products/92981/intel-xeon-processor-e5-2630-v4-25m-cache-2-20-ghz.html}
{aquí}.}, con las siguientes características:

\begin{itemize}[leftmargin=1.5cm, itemsep=1pt]
  \item \emph{CPU(s)}: 20
  \item \emph{Thread(s) per core}: 2
  \item \emph{Core(s) per socket}: 10
  \item \emph{Socket(s)}: 1
  \item \emph{CPU MHz}: 1200.221
  \item \emph{L1d cache}: 320 KiB
  \item \emph{L1i cache}: 320 KiB
  \item \emph{L2 cache}: 2.5 MiB
  \item \emph{L3 cache}: 25 MiB
\end{itemize}

La PC local tiene un \verb|Intel Core i7-7500U| con estas características:

\begin{itemize}[leftmargin=1.5cm, itemsep=1pt]
  \item \emph{CPU(s)}: 4
  \item \emph{Thread(s) per core}: 2
  \item \emph{Core(s) per socket}: 2
  \item \emph{Socket(s)}: 1
  \item \emph{CPU MHz}: 523.441
  \item \emph{L1d cache}: 32K
  \item \emph{L1i cache}: 32K
  \item \emph{L2 cache}: 256K
  \item \emph{L3 cache}: 4096K
\end{itemize}

Se probó en ambos lados con y sin el \emph{flag} de compilación \verb|-march=native|.
Esta bandera de compilación indica al compilador que habilite todos los
subconjuntos de instrucciones soportadas por la arquitectura nativa, lo cual
produce un código optimizado, pero únicamente para la máquina en la que se está
compilando el programa.

Seguidamente se muestran tablas con los resultados de las ejecuciones con y sin
la bandera tanto en el nodo como en la PC local.
Estas tablas contienen la cantidad de hilos con que se ejecutó el programa, el
tiempo que duró su ejecución y un factor de escala, que indica cómo va
escalando el tiempo de acuerdo a la cantidad de hilos utilizados.

\begin{table}[H]
  \centering
  \begin{tabular}{|| c | c | c ||} \hline
    Cantidad de hilos & Tiempo {[}s{]} & \emph{Scale Factor} \\ \hline
    1                 & 67.74          & -- \\ \hline
    2                 & 36.26          & 1.87   \\ \hline
    4                 & 19.98          & 3.39   \\ \hline
    8                 & 11.69          & 5.79   \\ \hline
    16                & 9.81           & 6.91   \\ \hline
    20                & 9.10           & \textcolor{OliveGreen}{7.44} \\ \hline
    32                & 9.14           & \textcolor{Red}{7.41}   \\ \hline
    64                & 8.95           & 7.57   \\ \hline
  \end{tabular}
  \caption{Duración del programa \textbf{sin} el \emph{flag} en el \emph{cluster}.}
  \label{test1n}
\end{table}


\begin{table}[H]
  \centering
  \begin{tabular}{|| c | c | c ||} \hline
    Cantidad de hilos & Tiempo {[}s{]} & \emph{Scale Factor}\\ \hline
    1                 & 47.76          & -- \\ \hline
    2                 & 25.14          & 1.90   \\ \hline
    4                 & 14.08          & 3.39   \\ \hline
    8                 & 8.14           & 5.87   \\ \hline
    16                & 6.48           & 7.37   \\ \hline
    20                & 5.90           & \textcolor{OliveGreen}{8.09}   \\ \hline
    32                & 5.95           & \textcolor{Red}{8.03}   \\ \hline
    64                & 6.01           & \textcolor{Red}{7.95}   \\ \hline
  \end{tabular}
  \caption{Duración del programa \textbf{con} el \emph{flag} en el \emph{cluster}.}
  \label{test2n}
\end{table}


\begin{table}[H]
  \centering
  \begin{tabular}{|| c | c | c ||} \hline
    Cantidad de hilos & Tiempo {[}s{]} & \emph{Scale Factor}\\ \hline
    1                 & 72.15          & -- \\ \hline
    2                 & 47.34          & 1.52   \\ \hline
    4                 & 45.76          & 1.58   \\ \hline
    8                 & 41.10          & \textcolor{OliveGreen}{1.76}   \\ \hline
    16                & 41.07          & \textcolor{Red}{1.76}   \\ \hline
    32                & 40.82          & 1.77   \\ \hline
    64                & 41.72          & \textcolor{Red}{1.73}   \\ \hline
  \end{tabular}
  \caption{Duración del programa \textbf{sin} el \emph{flag} en la PC.}
  \label{test1p}
\end{table}


\begin{table}[H]
  \centering
  \begin{tabular}{|| c | c | c ||} \hline
    Cantidad de hilos & Tiempo {[}s{]} & \emph{Scale Factor}\\ \hline
    1                 & 41.54          & -- \\ \hline
    2                 & 23.66          & 1.75   \\ \hline
    4                 & 19.93          & 2.08   \\ \hline
    8                 & 19.83          & \textcolor{OliveGreen}{2.09}   \\ \hline
    16                & 19.86          & \textcolor{Red}{2.09}   \\ \hline
    32                & 19.90          & \textcolor{Red}{2.09}   \\ \hline
    64                & 19.87          & \textcolor{Red}{2.09}   \\ \hline
  \end{tabular}
  \caption{Duración del programa \textbf{con} el \emph{flag} en la PC.}
  \label{test2p}
\end{table}

\subsubsection{Análisis de los tiempos}
\label{analisis}
Viendo los los resultados de las tablas anteriores, se nota claramente que al
utilizar la bandera de compilación \verb|-march=native|, mejoran notablemente
los tiempos de ejecución y el factor de escala, ya que en las tablas
\textbf{\ref{test2n}} y \textbf{\ref{test2p}} los factores de escala se
comportan de manera adecuada, es decir, las ejecuciones con una cantidad de
hilos mayor a los físicos, causan tiempos de ejecución mayores debido a que se
realizan más ciclos de reloj a causa de los cambios de contexto y sincronización
de las caché.

En las tablas \textbf{\ref{test1n}} y \textbf{\ref{test1p}}, sin embargo, lo
dicho anteriormente no ocurre. En el caso del nodo cuando se realiza una
ejecución con 64 hilos, el factor de escala vuelve a aumentar en lugar de
disminuir, mientras que en el caso de la PC, cuando llega a 32 ejecuciones el
factor aumenta y en la ejecución siguiente vuelve a disminuir. Claramente la
optimización hecha por \verb|-march=native| realiza una corrección de este
comportamiento.

Cabe mencionar que antes de estas ejecuciones se realizaron pruebas con distintas
banderas de compilación, como \verb|-O2| y \verb|-O1|, que aumentaron el tiempo
de ejecución del programa. Debido a esto, el programa ejecutado en el cluster es
el que mejor rendimiento obtuvo en la PC local.

\section{Conclusiones}
\label{conc}
Los requerimientos pedidos por la cátedra fueron cumplidos con éxito: el programa
realiza la aplicación de los filtros dentro y fuera de la circunferencia del radio
pasado como parámetro mediante paralelización con \emph{OpenMP}.


\begin{thebibliography}{9}
\bibitem{bmpfor}
\emph{Wikipedia, BMP file format},
\\\texttt{\href{https://en.wikipedia.org/wiki/BMP_file_format}{Link to Wikipedia}}

\bibitem{omp}
\emph{OpenMP},
\\\texttt{\href{https://www.openmp.org/}{Link to OpenMP}}

\bibitem{global}
\emph{Nora Sandler: C compiler, Part 10},
\\\texttt{\href{https://norasandler.com/2019/02/18/Write-a-Compiler-10.html}{Link to Global variables}}

\bibitem{perfw}
\emph{Wikipedia: perf},
\\\texttt{\href{https://en.wikipedia.org/wiki/Perf_(Linux)}{Link to perf}}

\bibitem{gprofw}
\emph{Linux Manual: gprof},
\\\texttt{\href{https://linux.die.net/man/1/gprof}{Link to gprof}}

\bibitem{valpage}
\emph{Valgrind},
\\\texttt{\href{http://valgrind.org/}{Link to Valgrind}}
\end{thebibliography}
\end{document}
