\documentclass[12pt,a4paper]{article}
\usepackage[utf8]{inputenc}
\usepackage[spanish]{babel} % Para modificar labels por defecto
\usepackage[bottom]{footmisc} % Para las notas al pie de la página
\usepackage[T1]{fontenc}
\usepackage{siunitx}
\usepackage{tocbibind} % Bibliografía en el indice
\usepackage{titlesec} % Posibilidad de editar los formatos de chapter
\usepackage{amsmath,amssymb,mathrsfs} % Matemáticas varias
\usepackage{tabularx} % Para las tablas
\usepackage[title]{appendix} % Para anexos
\usepackage{fancyhdr}
\renewcommand{\baselinestretch}{1} % Interlineado. 1 es estandar
% --- Arreglos varios para la inclusion de imagenes
\usepackage{float}
\usepackage[pdftex]{graphicx}
\usepackage{subfigure}
\graphicspath{{/home/mrgreen/Pictures/}}
\usepackage[usenames,dvipsnames]{color}
\DeclareGraphicsExtensions{.png,.jpg,.pdf,.mps,.gif,.bmp}
% --- Para las dimensiones de los márgenes etc
\frenchspacing \addtolength{\hoffset}{-1.5cm}
\addtolength{\textwidth}{3cm} \addtolength{\voffset}{-2.5cm}
\addtolength{\textheight}{4cm}
% --- Para el encabezado
\setlength{\headheight}{33pt}

\fancyhead[R]{\includegraphics[height=1cm]{logo_fcefyn_nuevo.jpg}}
\fancyhead[L]{\includegraphics[height=1cm]{unc1_a.jpg}}
\fancyhead[C]{}
\fancyfoot[C]{Página \thepage} \renewcommand{\footrulewidth}{0.4pt}
\pagestyle{fancy}
% --- Para las tablas
\usepackage{multirow} % Juntar filas
\newcolumntype{L}[1]{>{\raggedright\arraybackslash}p{#1}} % Justificar Izq
\newcolumntype{C}[1]{>{\centering\arraybackslash}p{#1}} % Justificar Centrar
\newcolumntype{R}[1]{>{\raggedleft\arraybackslash}p{#1}} % Justificar Der
\usepackage[numbered]{bookmark} % Para que figure las secciones en el PDF
\usepackage{listings} % Para poner código
\usepackage{enumitem} % Para editar las propiedades de los items
\lstset{frame=single} % Código en un cuadro
%\definecolor{negrou}{gray}{0.15} % Color gris para el fondo de las terminales
\lstset{basicstyle=\ttfamily} % Código en un cuadro
% --- Para Anexo
\addto\captionsspanish{%
  \renewcommand\appendixname{ANEXO}
  \renewcommand\appendixpagename{ANEXOS}
  \renewcommand\tablename{Tabla}
}
% --- Quita los cuadrados de colores en los hipervínculos
\hypersetup{
	colorlinks = true,
	linkcolor  = black,
	urlcolor   = black,
  citecolor  = black,
  filecolor  = black,
  anchorcolor = black}
% -------------------------------------------------------- %
% Definicion de colores para el codigo

\lstdefinelanguage{Ce}{
  backgroundcolor=\color{white},		%color de fondo
  commentstyle=\color{CadetBlue},		%color de los comentarios
  keywordstyle=\color{Plum},			%color de las palabras reservadas
  numberstyle=\tiny\color{ForestGreen},	%numero de línea con tamaño y color
  numbers=left,                    	%posicion de los numeros
  numbersep=5pt,   					%espaciado entre los numeros
  stringstyle=\color{OliveGreen},		%color de las strings
  basicstyle=\ttfamily\footnotesize,	%estilo de fuente
  breakatwhitespace=false,         	%romper estacios en blanco
  breaklines=true,                 	%romper lineas
  frame=single,						%para encuadrar el codigo
  rulecolor=\color{black},			%color del cuadro
  showspaces=false,
  showstringspaces=false,
  showtabs=false,
  tabsize=2,
  language=C
}

\renewcommand{\lstlistingname}{Código}

% -------------------------------------------------------- %

\begin{document}

\begin{titlepage}
    \begin{center}
      \vspace*{1cm}

      \vspace{2cm}
      \includegraphics[width=0.4\textwidth]{unc1_a.jpg}
      \includegraphics[width=0.4\textwidth]{logo_fcefyn_nuevo.jpg}

      \Huge
      \textbf{Cátedra de Sistemas Operativos II}

      \vspace{3.5cm}

      \textbf{Trabajo Práctico N\si{\degree} III}

      \vfill

      \vspace{0.8cm}



      \Large
      Piñero, Tomás Santiago\\
      4 de Junio de 2020
    \end{center}
\end{titlepage}
% -------------------------------------------------------- %
% --- Tabla de contenidos

\setcounter{secnumdepth}{3}
\setcounter{tocdepth}{5}
\tableofcontents

% -------------------------------------------------------- %

\newpage
\renewcommand{\baselinestretch}{1}
\setlength{\parskip}{0.5em}

\section{Introducción}
\label{intro}

\subsection{Objetivo}
\label{objetivo}
El objetivo del trabajo es lograr una visión \emph{end to end} de una
implementación básica de una \emph{RESTful API} sobre un sistema embedido.


\subsection{Definiciones, Acrónimos y Abreviaturas}
\label{daa}

\begin{itemize}
  \item \emph{API}: \emph{Application Programming Interface}. Es un conjunto
  de funciones ofrecidas para ser utilizadas por otro software.
  \item \emph{REST}: \emph{Representational State Transfer}. Es un diseño de
  software para los servicios web. Estos servicios web son llamados
  \emph{RESTful Web services} y proveen operaciones entre computadoras en Internet.
  \cite{restful}
  \item \emph{Framework}: es una estructura conceptual y tecnológica de soporte
  definido, normalmente con módulos de software concretos, que puede servir de
  base para la organización y desarrollo de software.\cite{framework}
\end{itemize}


\section{Descripción general}
\label{desc}

\subsection{Restricciones}
\label{restrictions}
A las restricciones dadas (ver \href{run:../Enunciado.pdf}
{\texttt{Enunciado.pdf}}) se le agregaron las siguientes:

\begin{enumerate}[leftmargin=1.5cm]
  \item El servidor en el que se instalen los servicios deben tener
  instaladas las dependencias necesarias:

  \begin{itemize}[leftmargin=1.5cm]
    \item \emph{Ulfius library}\cite{ulfius}.
    \item \emph{Nginx}\cite{nginx}.
  \end{itemize}

  \item El usuario con el cual que se ejecuten los servicios, \verb|usertp3|, ya
  debe existir en el servidor. El \emph{Makefile} no se encarga de
  crear dicho usuario.
  \item El usuario mencionado en el ítem anterior debe tener permisos para
  ejecutar el comando \verb|useradd| sin la petición de contraseña.
  \item Debe existir el archivo \verb|/etc/.nginxpasswd|. Este archivo es el que
  \emph{Nginx} utiliza para permitir el acceso a los servicios.
  \item El ciente debe tener en \verb|/etc/hosts| la dirección IP del servidor
  asociada al dominio \verb|tp3.com|.
\end{enumerate}


\subsection{Esquema del proyecto}
\label{esq}

El proyecto está dividido en las siguientes carpetas:

\begin{itemize}[leftmargin=1.5cm]
  \item \textbf{bin}: contiene los archivos ejecutables de los servicios.
  \item \textbf{inc}: contiene el \emph{header `utilities.h'}, que contiene las
  funciones útiles para los servicios: escribir el \emph{log} de eventos y la
  obtención del \emph{timestamp}.
  \item \textbf{config}: contiene los archivos de configuración para \emph{Nginx}
  y los servicios para \emph{systemd}.
  \item \textbf{src}: contiene los códigos fuente.

  \begin{itemize}[leftmargin=1cm, nosep]
    \item \emph{status\_service.c}: crea el servicio de estado en el puerto \verb|8081|.
    \item \emph{users\_service.c}: crea el servicio de usuarios en el puerto \verb|8080|.
    \item \emph{utilities.c}: implementa las funciones del \emph{utilities.h}.
  \end{itemize}
\end{itemize}

\subsection{Requisitos futuros}
\label{futurereq}

\begin{itemize}[leftmargin=1.5cm]
  \item Crear una página que consuma los datos devueltos por los servicios.
\end{itemize}


\section{Diseño de solución}
\label{solucion}
Como se mencionó en la sección anterior, se organizó el proyecto en varias
carpetas (ver \textbf{\ref{esq}}).

Se decidió crear dos archivos \verb|.c|, uno para cada servicio.
``\emph{users\_service}'' se encargará de proveer los servicios de \verb|/api/users|
y ``\emph{status\_service}'' proveerá los servicios de \\
\verb|/api/servers/hardwareinfo|.

Para realizar el archivo \emph{log}, se creó \verb|utilities.c|, que contiene la
función para escribir el archivo y otras funcionalidades que sean compartidas
entre los servicios.

\subsection{\emph{Makefile}}
\label{make}
Las recetas disponibles en el \emph{Makefile} son las siguientes:

\begin{enumerate}[leftmargin=1.5cm]
  \item \emph{all}: genera el ejecutable del sistema.
  \item \emph{check}: corre \emph{cppcheck} sobre el directorio
  \verb|src|.
  \item \emph{doc}: genera la documentación del código en HTML
  utilizando \emph{doxygen} en la carpeta llamada \verb|Doc| y la
  abre en el navegador \emph{Firefox}.\footnote{Si no se encuentra
  instalado, se sugiere cambiar el navegador en la receta.}
  \item \emph{install}: se encarga de copiar los archivos de configuración de
  \emph{nginx} y de los servicios para \emph{systemd} a los directorios
  correspondientes.
  \item \emph{clean}: elimina los directorios \verb|bin| y
  \verb|Doc|.
\end{enumerate}


\section{Implementación y resultados}
\label{impres}

\subsection{Servicios}
\label{servicios}
En un principio se utilizó el archivo \emph{helloworld} ofrecido por \emph{Ulfius}
para comprender el funcionamiento del \emph{framework}.

Posteriormente, se crearon los archivos mencionados en la sección anterior, uno
para cada serivicio, con sus \emph{endpoints} correspondientes. Cada uno de
éstos consistía simplemente en devolver un mensaje de tipo \emph{string} para
corroborar el funcionamiento.

\subsubsection{\emph{Users}}
\label{users}
Para obtener el listado de usuarios, se implementó la funcón \emph{get\_users()},
que devuelve un objeto \emph{JSON} con la información de los usuarios.

Esta función es utilizada en el \emph{callback\_users\_get}, que se
ejecuta cada vez que el servicio recibe una \emph{request HTTP} mediante el
método \emph{GET}.

\begin{lstlisting}[caption={Función `\emph{get\_users()}'.}, label={get}, language=Ce]
json_t *get_users()
{
  int users = 0;
  json_t *json_users_object = json_object();
  json_t *json_users_array  = json_array();

  json_object_set_new(json_users_object, "data", json_users_array);

  struct passwd *passwd_info = getpwent();

  while(passwd_info)
    {
      users++;
      json_t *json_array_data;
      json_array_data = json_pack("{s:i, s:s}", "user_id", passwd_info->pw_uid,
                                  "username", passwd_info->pw_name);

      json_array_append(json_users_array, json_array_data);

      passwd_info = getpwent();
    }

  endpwent();

  syslog(LOG_NOTICE, "Usuarios listados: %d\n", users);
  char msg[BUFF];
  sprintf(msg, "Usuarios listados: %d\n", users);
  write_log(get_date(), USER, msg);

  return json_users_object;
}
\end{lstlisting}

\newpage
En cuanto al método para agregar un usuario nuevo, lo primero que se realiza al
recibir una petición, es obtener un objeto \emph{JSON} de la misma y verificar
que el formato es el correcto. De no serlo, se evnía una respuesta con código
400 y una descripción ``\emph{Bad request}". Caso contrario, se revisa si el
usuario ya existe en el sistema, sino existe, se lo crea y se devuelve la información.

\begin{lstlisting}[caption={\emph{Endpoint /users, POST method.}}, label={post}, language=Ce]
int callback_users_post(const struct _u_request * request,
                        struct _u_response * response, void * user_data)
{
  json_t *json_request = ulfius_get_json_body_request(request, NULL);
  json_t *body;

  const char *user   = json_string_value(json_object_get(json_request, "username"));
  const char *passwd = json_string_value(json_object_get(json_request, "password"));

  if(user == NULL || passwd == NULL)
    {
      body = json_pack("{s:s}", "description", "Bad request");
      ulfius_set_json_body_response(response, 400, body);
      return U_CALLBACK_CONTINUE;
    }

  struct passwd *info = getpwnam(user);
  if(info != NULL)
    {
      body = json_pack("{s:s}", "description", "El usuario ya existe");
      ulfius_set_json_body_response(response, 409, body);
      return U_CALLBACK_CONTINUE;
    }

  char cmd[BUFF];
  sprintf(cmd, "sudo useradd %s -p $(openssl passwd -5 '%s')", user, passwd);
  char *tmp = get_popen(cmd);
  free(tmp);

  char *timestamp = get_date();

  struct passwd *passwd_info = getpwnam(user);

  body = json_pack("{s:i, s:s, s:s}", "id", passwd_info->pw_uid,
                   "username", user, "created_at", timestamp);

  ulfius_set_json_body_response(response, 200, body);

  syslog(LOG_NOTICE, "Usuario %d creado", passwd_info->pw_uid);

  char msg[BUFF];
  sprintf(msg,"Usuario %d creado\n", passwd_info->pw_uid);
  write_log(timestamp, USER, msg);
  free(timestamp);

  return U_CALLBACK_CONTINUE;
}
\end{lstlisting}

\subsubsection{\emph{Status}}
\label{status}
La obtención de los datos pedidos se guardan un una estructura llamada
\emph{hwinfo}. Esta, a su vez, contiene un dato tipo \emph{upinfo}, que es otra
estructura que contiene los datos que indican el tiempo transcurrido desde que
se inició el sistema.

\begin{lstlisting}[caption={Estructura `\emph{upinfo}'.}, label={up}, language=Ce]
typedef struct uptime   /** Estructura para obtener el uptime. */
{
  int time; /** Uptime total en segundos. */
  int h;    /** Horas de uptime. */
  int min;  /** Minutos de uptime. */
  int seg;  /** Segundos de uptime. */
} upinfo;
\end{lstlisting}

\begin{lstlisting}[caption={Estructura `\emph{hwinfo}'.}, label={hwinfo}, language=Ce]
typedef struct hardinfo /** Estructura para obtener la informacion del hardware. */
{
  char *kernelVersion; /** Version de kernel. */
  char *processorName; /** Procesador. */
  int totalCPUCore;    /** Cantidad de nucleos del procesador. */
  int totalMemory;     /** Cantidad de memoria RAM total en MB. */
  int freeMemory;      /** Cantidad de memoria RAM libre en MB. */
  int diskTotal;       /** Capacidad del disco en GB. */
  int diskFree;        /** Espacio libre del disco en GB. */
  float loadAvg;       /** Carga promedio del sistema en 1 minuto. */
  upinfo uptime;       /** Cuanto tiempo hace que se inicio el sistema. */
} hwinfo;
\end{lstlisting}

La función `\emph{get\_hwinfo}' se encarga de completar estas estructuras, que
son utilizados en la función \emph{callback\_status}, donde se arma un objeto
\emph{JSON} con esta estructura.

\begin{lstlisting}[caption={Función `\emph{get\_hwinfo}'.}, label={gethw}, language=Ce]
void get_hwinfo(hwinfo *hard)
{
  hard->kernelVersion = get_popen("uname -r");

  char *tmp = get_popen("cat /proc/loadavg");
  sscanf(tmp, "%f", &hard->loadAvg);

  tmp = get_popen("cat /proc/uptime");
  sscanf(tmp, "%d", &hard->uptime.time);

  hard->uptime.h   = hard->uptime.time/60/60%24;
  hard->uptime.min = hard->uptime.time/60%60;
  hard->uptime.seg = hard->uptime.time%60;

  tmp = get_popen("free --mega | grep Mem");
  sscanf(tmp, "%*s %d %*d %d", &hard->totalMemory, &hard->freeMemory);

  tmp = get_popen("lscpu | grep name");
  tmp = remove_spaces(tmp);
  hard->processorName = tmp;

  tmp = get_popen("lscpu | grep socket");
  tmp = remove_spaces(tmp);
  sscanf(tmp, "%d", &hard->totalCPUCore);

  tmp = get_popen("df . | grep /");
  sscanf(tmp, "%*s %d %*d %d", &hard->diskTotal, &hard->diskFree);
  hard->diskTotal /= 1024000;
  hard->diskFree /= 1024000;

  free(tmp);
}
\end{lstlisting}

\subsection{\emph{Nginx}}
\label{secnginx}
Para la configuración de \emph{nginx}, se siguió la documentación provista por
los desarrolladores\cite{ndoc}, creando un archivo en el directorio
\verb|/etc/nginx/sites-available| llamado \verb|tp3so2|.

\begin{lstlisting}[caption={Configuración del sitio \texttt{tp3.com.}}, label={tp3}]
server {
        listen :80;

        server_name tp3.com;

        auth_basic           "TP3 Area";
        auth_basic_user_file /etc/.nginxpasswd;

        location /api/users {
                proxy_pass http://localhost:8080/users;
        }

        location /api/servers/hardwareinfo {
                proxy_pass http://localhost:8081/hardwareinfo;
                proxy_set_header X-Real-IP $remote_addr;
        }
}
\end{lstlisting}

El Código \textbf{\ref{tp3}} establece el nombre del servidor, el nombre del
área protegida a la cual únicamente los usuarios registrados en
\verb|/etc/.nginxpasswd| tienen acceso y los \emph{locations} del servidor.

\newpage
\subsection{\emph{systemd}}
\label{secsystemd}
Por último, con \emph{nginx} y los binarios de los servicios funcionando
correctamente se procedió a crear los archivos necesarios para habilitarlos como
servicios. El archivo \verb|.service| tiene la siguiente forma:

\begin{lstlisting}[caption={Estructura del servicio.}, label={service}]
[Unit]
Description=[Descripcion del servicio]
Requires=nginx.service
After=nginx.service

[Service]
Type=simple
ExecStart=/usr/bin/[ejecutable_servicio]
User=usertp3

[Install]
WantedBy=multi-user.target
\end{lstlisting}

Los dos archivos correspondientes a los servicios se ubican en el directorio
\verb|/etc/systemd/system/|.

Además del \emph{log} que se pidió, se realiza un \emph{log} en \emph{journal}
mediante la función \emph{syslog()}. Esto se realizó para obtener tener una
copia en otro directorio del sistema, \verb|/var/log/| y cumplir con el formato
requerido por la cátedra.

\subsection{Resultados}
\label{results}
El sistema se encuentra corriendo en una \emph{Raspberry Pi 3B+} conectada a la
misma red WiFi.

La siguiente figura muestra el resultado de las pruebas implementadas en
\emph{Postman}.

\begin{figure}[H]
  \centering
  \includegraphics[scale=0.4]{results.png}
  \caption{Resultados de los \emph{tests} en \emph{Postman}.}
  \label{postman}
\end{figure}

Como se puede ver, el tiempo menor a 200ms de requerimiento para el método de
agregar un usuario nuevo no se cumple. Se realizaron varios cambios en la
implementación y el tiempo mínimo que se pudo lograr fue de 300ms.

\begin{figure}[H]
  \centering
  \includegraphics[scale=0.8]{hwinfo.png}
  \caption{Información del hardware del servidor.}
  \label{raspinfo}
\end{figure}

La Fig. \textbf{\ref{raspinfo}} muestra el resultado de la consulta a
\emph{hardwareinfo}. Los valores correspondientes a la memoria se muestran en
[MB] y los valores de disco en [GB].

\section{Conclusiones}
\label{conc}
El trabajo práctico no fue complicado de implementar, a diferencia de los
anteriores.

El único requerimiento que no se pudo cumplir es el del tiempo de respuesta de
la solicitud para agregar el usuario, que puede deberse tanto a detalles de la
implementación como de la latencia en la red.


\begin{thebibliography}{9}
\bibitem{restful}
\emph{Wikipedia, REST},
\\\texttt{\href{https://en.wikipedia.org/wiki/Representational_state_transfer}{Link to Wikipedia}}

\bibitem{framework}
\emph{Wikipedia, Framework},
\\\texttt{\href{https://en.wikipedia.org/wiki/Software_framework}{Link to Wikipedia}}

\bibitem{ulfius}
\emph{Github, Ulfius},
\\\texttt{\href{https://github.com/babelouest/ulfius}{Link to Ulfius}}

\bibitem{nginx}
\emph{Nginx},
\\\texttt{\href{https://nginx.org/en/}{Link to Nginx}}

\bibitem{ndoc}
\emph{Nginx, Documentation},
\\\texttt{\href{https://docs.nginx.com/}{Link to Nginx documentation}}

\end{thebibliography}
\end{document}
