\documentclass[12pt,a4paper]{article}
\usepackage[utf8]{inputenc}
\usepackage[spanish]{babel} % Para modificar labels por defecto
\usepackage[bottom]{footmisc} % Para las notas al pie de la página
\usepackage[T1]{fontenc}
\usepackage{siunitx}
\usepackage{tocbibind} % Bibliografía en el indice
\usepackage{titlesec} % Posibilidad de editar los formatos de chapter
\usepackage{amsmath,amssymb,mathrsfs} % Matemáticas varias
\usepackage{tabularx} % Para las tablas
\usepackage[title]{appendix} % Para anexos
\usepackage{fancyhdr}
\renewcommand{\baselinestretch}{1} % Interlineado. 1 es estandar
% --- Arreglos varios para la inclusion de imagenes
\usepackage{float}
\usepackage[pdftex]{graphicx}
\usepackage{subfigure}
\graphicspath{{/home/mrgreen/Pictures/}}
\usepackage[usenames,dvipsnames]{color}
\DeclareGraphicsExtensions{.png,.jpg,.pdf,.mps,.gif,.bmp}
% --- Para las dimensiones de los márgenes etc
\frenchspacing \addtolength{\hoffset}{-1.5cm}
\addtolength{\textwidth}{3cm} \addtolength{\voffset}{-2.5cm}
\addtolength{\textheight}{4cm}
% --- Para el encabezado
\setlength{\headheight}{33pt}

\fancyhead[R]{\includegraphics[height=1cm]{logo_fcefyn_nuevo.jpg}}\fancyhead[L]{\includegraphics[height=1cm]{unc1_a.jpg}}\fancyhead[C]{} \fancyfoot[C]{Página \thepage} \renewcommand{\footrulewidth}{0.4pt}
\pagestyle{fancy}
% --- Para las tablas
\usepackage{multirow} % Juntar filas
\newcolumntype{L}[1]{>{\raggedright\arraybackslash}p{#1}} % Justificar Izq
\newcolumntype{C}[1]{>{\centering\arraybackslash}p{#1}} % Justificar Centrar
\newcolumntype{R}[1]{>{\raggedleft\arraybackslash}p{#1}} % Justificar Der
\usepackage[numbered]{bookmark} % Para que figure las secciones en el PDF
\usepackage{listings} % Para poner código
\usepackage{enumitem} % Para editar las propiedades de los items
\lstset{frame=single} % Código en un cuadro
% --- Para Anexo
\addto\captionsspanish{%
  \renewcommand\appendixname{ANEXO}
  \renewcommand\appendixpagename{ANEXOS}
}
% --- Quita los cuadrados de colores en los hipervínculos
\hypersetup{
	colorlinks = true,
	linkcolor  = black,
	urlcolor   = black,
  citecolor  = black,
  filecolor  = black,
  anchorcolor = black}
% -------------------------------------------------------- %
% Definicion de colores para el codigo

\lstdefinelanguage{Ce}{
  backgroundcolor=\color{white},		%color de fondo
  commentstyle=\color{CadetBlue},		%color de los comentarios
  keywordstyle=\color{Plum},			%color de las palabras reservadas
  numberstyle=\tiny\color{ForestGreen},	%numero de línea con tamaño y color
  numbers=left,                    	%posicion de los numeros
  numbersep=5pt,   					%espaciado entre los numeros
  stringstyle=\color{OliveGreen},		%color de las strings
  basicstyle=\ttfamily\footnotesize,	%estilo de fuente
  breakatwhitespace=false,         	%romper estacios en blanco
  breaklines=true,                 	%romper lineas
  frame=single,						%para encuadrar el codigo
  rulecolor=\color{black},			%color del cuadro
  showspaces=false,
  showstringspaces=false,
  showtabs=false,
  tabsize=2,
  language=C
}

\renewcommand{\lstlistingname}{Código}

% -------------------------------------------------------- %

\begin{document}

\begin{titlepage}
    \begin{center}
      \vspace*{1cm}

      \vspace{2cm}
      \includegraphics[width=0.4\textwidth]{unc1_a.jpg}
      \includegraphics[width=0.4\textwidth]{logo_fcefyn_nuevo.jpg}

      \Huge
      \textbf{Cátedra de Sistemas Operativos II}

      \vspace{3.5cm}

      \textbf{Trabajo Práctico N\si{\degree} I}

      \vfill

      \vspace{0.8cm}



      \Large
      Piñero, Tomás Santiago\\
      XX de Abril de 2020
    \end{center}
\end{titlepage}
% -------------------------------------------------------- %
% --- Tabla de contenidos

\setcounter{secnumdepth}{1}
\setcounter{tocdepth}{5}
\tableofcontents

% -------------------------------------------------------- %

\newpage
\renewcommand{\baselinestretch}{1}
\setlength{\parskip}{0.5em}

\section{Introducción}
\label{intro}

\subsection{Objetivo}
\label{objetivo}
El objetivo del trabajo es diseñar e implementar un software que haga uso de las
API de IPC del Sistema Operativo, implementando los conocimientos adquiridos en
las clases.

\subsection{Arquitectura del Sistema}
\label{sistema}
El sistema está formado por dos \emph{Hosts}. El \emph{Host 1} es el cliente,
que se conecta al servidor (\emph{Host 2}) a un puerto fijo.

El servidor está conformado por tres procesos, cada uno con funciones específicas.

\begin{enumerate}[leftmargin=1.5cm]
  \item \textbf{\emph{Auth service}}: encargado de proveer las funcionalidades relacionadas
  al manejo de la base de dato de los usuarios:

  \begin{itemize}[leftmargin=1cm, nosep]
    \item Validar de usuario.
    \item Listar usuarios.
    \item Cambiar contraseña del usuario.
  \end{itemize}

  \item \textbf{\emph{File service}}: encargado de proveer las funcionalidades relacionadas
  con las imágenes desponibles para descarga:

  \begin{itemize}[leftmargin=1cm, nosep]
    \item Mostrar información de las imágenes disponibles.
    \item Establecer la transferencia de la imagen a través de un nuevo
    \emph{socket}.
  \end{itemize}

  \item \textbf{\emph{Primary server}}: encargado de establecer las conexiones y recibir
  los comandos del cliente. Hace uso de los dos procesos anteriores.
\end{enumerate}

\begin{figure}[H]
  \centering
  \includegraphics[scale=0.6]{arqtp1.png}
  \caption{Arquitectura del sistema.}
  \label{arqui}
\end{figure}

\newpage

\subsection{Definiciones, Acrónimos y Abreviaturas}
\label{daa}

\begin{itemize}
  \item \emph{API}: \emph{Application Programming Interface}. Es un conjunto
  de funciones ofrecidas para ser utilizadas por otro software.
  \item \emph{IPC}: \emph{Inter-Process Communication}. Mecanismo del Sistema
  Operativo que permite a los procesos comunicarse y sincronizarse entre sí.
  \item \emph{TCP/IP}: conjunto de protocolos que posibilitan las comunicaciones
  de Internet.
\end{itemize}


\section{Descripción general}
\label{desc}

\subsection{Restricciones}
\label{restrictions}
A las restricciones dadas (ver \href{run:../Enunciado.pdf}
{\texttt{Enunciado.pdf}}) se le agregaron las siguientes:

\begin{enumerate}[leftmargin=1.5cm]
  \item El usuario y contraseña no pueden ser los mismos.
  \item El comando para descargar la imagen debe tener el
  siguiente formato:

  \begin{center}
    \verb|file down `nombre_imagen' `directorio_usb'|
  \end{center}

  \item Las imágenes para descargar deben estar ubicadas en la
  carpeta \verb|imgs|.
\end{enumerate}


\subsection{Esquema del proyecto}
\label{esq}

El proyecto está dividido en las siguientes carpetas:

\begin{itemize}[leftmargin=1.5cm]
  \item \textbf{bin}: contiene los archivos ejecutables.
  \item \textbf{inc}: contiene los \emph{headers} utilizados por los
  códigos fuente:

  \begin{itemize}[leftmargin=1cm, nosep]
    \item \emph{messages.h}: \emph{header} con los datos para el
    envío y recepción de mensajes entre los procesos que conforman
    el servidor.
    \item \emph{sockets.h}: \emph{header} con los datos y funciones
    para la utilización de los \emph{sockets}: su creación y su
    lectura/escritura.
  \end{itemize}

  \item \textbf{imgs}: contiene las imágenes disponibles para descargar.
  \item \textbf{res}: contiene la base de datos de los usuarios.
  \item \textbf{src}: contiene los códigos fuente.

  \begin{itemize}[leftmargin=1cm, nosep]
    \item \emph{client.c}
    \item \emph{primary\_server.c}
    \item \emph{auth\_service.c}
    \item \emph{files\_service.c}
    \item \emph{sockets.c}
  \end{itemize}

\end{itemize}

\subsection{Requisitos futuros}
\label{futurereq}

\begin{itemize}[leftmargin=1.5cm]
  \item Utilizar \emph{CMake} para generar el \emph{Makefile}.
  \item Lograr que el comando \emph{exit} no termine la conexión del servidor
\end{itemize}

% \newpage
\section{Diseño de solución}
\label{solucion}

\subsection{Comunicación entre los procesos}
\label{ipc}
Además de la implementación de \emph{sockets}, se debe elegir
otro mecanismo de los IPC existentes:

\begin{itemize}[leftmargin=1.5cm]
  \item \emph{PIPEs} y \emph{FIFOs}.
  \item Semáforos.
  \item \emph{Signals}.
  \item Memoria compartida.
  \item Cola de mensajes.
\end{itemize}

Para realizar la comunicación entre los procesos del servidor se
decidió utilzar el sistema de cola de mensajes \emph{SystemV}, ya
que múltiples procesos pueden acceder a una misma cola de mensajes
sin problemas de concurrencia.

La cola de mensajes es \emph{única} para los tres procesos que
conforman el servidor y es creada por el proceso \emph{server}.
Los estructura de mensaje utilizada para la cola es la siguiente:

\begin{lstlisting}[caption={Estructura de mensaje.}, label={msg}, language=Ce]
struct msg            /* Estructura para mensajes generales. */
{
  long mtype;         /* Identificador del mensaje. */
  char msg[STR_LEN];  /* Contenido del mensaje. */
};
\end{lstlisting}

El identificador de mensaje que se muestra en el Código \ref{msg}
puede tomar los siguientes valores:

\begin{itemize}[leftmargin=1.5cm]
  \item \textbf{to\_prim}: los mensajes dirigidos al proceso
  \emph{server}.
  \item \textbf{to\_auth}: los mensajes dirigidos al proceso
  \emph{auth}.
  \item \textbf{to\_file}: los mensajes dirigidos al proceso
  \emph{fileserv}.
\end{itemize}

\newpage

\subsection{Procesos del servidor}
\label{server}

A continuación se explica brevemente el diseño para cada uno de los
procesos involucrados en el sistema\footnote{Para información más
detallada sobre el funcionamiento de cada proceso se recomienda ver
la documentación del mismo}.

\subsubsection{\emph{Server}}
\label{server_ps}
Es el intermediario entre el cliente y los otros dos procesos que
conforman el servidor (ver Figura \ref{arqui}).

Este proceso tiene como hijos al proceso \emph{auth} y \emph{fileserv}.
La creación de estos procesos se realiza una vez creados los medios
de comunicación necesarios (\emph{socket} y cola de mensajes). Se
decidió por esta implementación debido a que de esta forma los
procesos hijos podrán acceder a la cola de mensajes sin problemas.

El protocolo de envío de mensajes entre ellos se muestra en el
siguiente diagrama de secuencia.

\begin{figure}[H]
  \center
  \includegraphics[height=14cm]{protocolo.png}
  \caption{Diagrama de secuencia de la comunicación cliente-servidor.}
  \label{com_seq}
\end{figure}

\subsubsection{\emph{Auth}}
\label{auth_ps}
Proceso encargado del manejo de la base de datos de los usuarios y
de la validación de las credenciales enviadas por el usuario.

Se permiten tres intentos de inicio de sesión, al superarse este
límite, el usuario se bloquea y se terminan todos los procesos del
sistema. Si el inicio de sesión es exitoso, se modifica la fecha
y hora del último inicio de sesión a la fecha actual.

\textbf{Base de datos}

Para simular la base de datos se utiliza un archivo \verb|.txt| con
valores separados con coma. La estructura sus contenidos es la
siguiente:

\begin{center}
  \verb|ID,usuario,contraseña,estado,última_conexión|
\end{center}


\subsubsection{\emph{Fileserv}}
\label{files_ps}
Este proceso se encarga del manejo de las imágenes disponibles
para descargar, las cuales se encuentran en la carpeta \verb|imgs|
y también de enviar la imagen seleccionada por el usuario mediante
un nuevo socket.

\subsection{\emph{Makefile}}
\label{make}
Las recetas disponibles en el \emph{Makefile} son las siguientes:

\begin{enumerate}
  \item \emph{all}: genera los ejecutables del sistema.
  \item \emph{check}: corre \emph{cppcheck} sobre el directorio
  \verb|src|.
  \item \emph{doc}: genera la documentación del código en HTML
  utilizando \emph{doxygen} en la carpeta llamada \verb|Doc| y la
  abre en el navegador  \emph{Firefox}.\footnote{Si no se encuentra
  instalado, se sugiere cambiar el navegador en la receta.}
  \item \emph{client}: crea el directorio \verb|bin| y genera el
  ejecutable correspondiente al cliente.
  \item \emph{server}: general los ejecutables correspondientes al
  servidor.
  \item \emph{clean}: elimina los directorios \verb|bin| y
  \verb|Doc|.
\end{enumerate}


\section{Implementación y resultados}
\label{impres}

\subsection{\emph{Sockets}}
El método de comunicación de \emph{sockets} es utilizado por tres
procesos: \emph{client}, \emph{server} y \emph{fileserv}, por lo
que se decidió realizar un \emph{header} con las funciones que se
utilizan para la creación de los \emph{sockets} y su lectura/
escritura.

A continuación se muestra brevemente el contenido del \emph{header}.

\begin{lstlisting}[caption={\emph{Header} para el uso de \emph{sockets}.}, label={socket}, language=Ce]
enum ports        /** Puertos para la conexion de los sockets. */
{
  port_ps = 4444, /** Puerto para `primary_server'. */
  port_fi = 5555  /** Puerto para `files_service'. */
};

#define STR_LEN 1024  /** Largo de los strings */

/** Escribe en el socket deseado un mensaje. */
ssize_t send_cmd(int sockfd, void *cmd);

/** Lee el mensaje del socket deseado. */
ssize_t recv_cmd(int sockfd, void *cmd);

/** Crea el socket en el puerto pedido. */
int create_svsocket(char *ip, uint16_t port);

/** Conecta al cliente en el puerto del servidor. */
int create_clsocket(char *ip, uint16_t port);
\end{lstlisting}

\subsection{Mensajes}
\label{msg_h}
Como se explicó en la subsección \ref{ipc}, se utiliza una cola de mensajes
entre los tres procesos que conforman el servidor, por lo que dichos procesos
utilizan una librería en común: \emph{messages.h}.

\begin{lstlisting}[caption={\emph{Header} para el uso de la cola de mensajes.}, label={msgql}, language=Ce]
#define STR_LEN 1024 /**< Largo de los strings */

/* Enum con con los destinos de los mensajes a enviar en la cola de mensajes. */
enum msg_ID
{
  to_auth = 1, /* Destinado a `auth_service'. */
  to_file = 2, /* Destinado a `files_service'. */
  to_prim = 3  /* Destinado a `primary_server'. */
};

/* Enum con los estados posibles del usuario. */
enum status
{
  blocked = 1, /* El usuario esta bloqueado. */
  active  = 2, /* El usuario esta activo. */
  invalid = 3, /* Credenciales invalidas. */
  not_reg = 4  /* Es usuario no existe. */
};

#define QU_PATH  "./server" /* Archivo para crear la cola de mensajes. */
\end{lstlisting}

En este \emph{header} también se encuentra incluida la estructura de mensaje
mostrada en el Código \ref{msg}.

\subsection{Resultados}
\label{resultados}
A continuación se muestran los resultados obtenidos al correr el programa con
los comandos requeridos.

\subsubsection{Inicio de sesión}
\label{login}

\begin{figure}[H]
  \centering
  \includegraphics[scale=0.6]{login_suc.png}
  \caption{Inicio de sesión exitoso.}
  \label{log_suc}
\end{figure}

\begin{figure}[H]
  \centering
  \includegraphics[scale=0.6]{login_f.png}
  \caption{Inicio de sesión fallido,}
  \label{log_f}
\end{figure}

\subsubsection{Listado de usuarios}
\label{users}

\begin{figure}[H]
  \centering
  \includegraphics[scale=0.6]{userls.png}
  \caption{Comando \emph{user ls}.}
  \label{userls}
\end{figure}

\subsubsection{Listado de imágenes}
\label{files}

El comando demora en mostrarse debido a que el cálculo del \emph{hash} MD5 se
realiza en el momento de ejecución.

\begin{figure}[H]
  \centering
  \includegraphics[scale=0.6]{filels.png}
  \caption{Comando \emph{file ls}.}
  \label{filels}
\end{figure}

\subsubsection{Descarga de imagen}
\label{down}

\begin{figure}[H]
  \centering
  \includegraphics[scale=0.6]{programa.png}
  \caption{Comando \emph{file down `imagen' `usb'}.}
  \label{filedown}
\end{figure}

\begin{figure}[H]
  \centering
  \includegraphics[scale=0.6]{hexdump.png}
  \caption{Comando \emph{hexdump} en USB.}
  \label{dumpusb}
\end{figure}

\begin{figure}[H]
  \centering
  \includegraphics[scale=0.6]{dumpiso.png}
  \caption{Comando \emph{hexdump} en imagen.}
  \label{dumpiso}
\end{figure}

\section{Conclusiones}
\label{conc}
Inserte conclusión horrible aquí.

\newpage
\begin{thebibliography}{9}
\bibitem{beejs}
\emph{Beej's Guide to Unix IPC, Message Queues},
\\\texttt{\href{https://beej.us/guide/bgipc/html/multi/mq.html}{Link to Message Queues}}


\bibitem{replace}
\emph{Codeforwin: Replace text in a line},
\\\texttt{\href{https://codeforwin.org/2018/02/c-program-replace-specific-line-a-text-file.html}
{Link to Replace text}}


\bibitem{date}
\emph{Stackoverflow: How to get date and time},
\\\texttt{\href{https://stackoverflow.com/questions/1442116/how-to-get-the-date-and-time-values-in-a-c-program}
{Lonk to Date-time}}

\bibitem{seqdiag}
\emph{Web Sequence Diagram},
\\\texttt{\href{https://www.websequencediagrams.com/}{Web sequence diagrams}}

\bibitem{filelist}
\emph{Stackoverflow: How to list files in a directory},
\\\texttt{\href{https://stackoverflow.com/questions/4204666/how-to-list-files-in-a-directory-in-a-c-program}{Link to list files}}

\bibitem{md5}
\emph{Stackoverflow: How to calculate MD5},
\\\texttt{\href{https://stackoverflow.com/questions/10324611/how-to-calculate-the-md5-hash-of-a-large-file-in-c}{Link to calculate MD5}}

\bibitem{md5s}
\emph{Stackoverflow: How to show MD5 as a string},
\\\texttt{\href{https://stackoverflow.com/questions/7627723/how-to-create-a-md5-hash-of-a-string-in-c}{Link to MD5 to string}}
\end{thebibliography}
\end{document}
